\documentclass[12]{article}
\usepackage{hyperref}
\begin{document}


%lectures - MW 4:05-5:25, 

\begin{center}{\bf Physics 512 Syllabus, 2022 (preliminary)}
\end{center}
\vskip 0.1in 
This course focuses on equipping students with the tools
needed to solve problems in physics using computers.  The course will
be taught using python, and students are strongly encouraged to have
at least some familiarity with basic python coding by the beginning of
the course.  Assessment is via hands-on problem sets (40\%), a final
coding project (30\%), and a take-home exam (30\%).  There will be an
option to not do the final coding project, in which case the grade
will be 60\% problem sets and 40\% exam.  
{\textit{Numerical Recipes, 3rd edition}} covers many of the course
subjects and it is strongly recommended.  Lectures will consiste of a
mix of blackboard derivations and live coding examples demonstrating
the principles in action.  We anticipate tutorial sessions will be
carried out in-person, and will focus on helping students implement
class concepts into working code.  Students are encouraged to provide
examples of computing challenges they face that can be used as practical examples. 
\vskip 0.1in
\noindent Lectures will be held in Rutherford 114 from 2:35 AM to 3:55
PM on Mondays and Wednesdays. Office hours and tutorials will be
scheduled in consultation with the class/TAs. 
I will do my best to
stream/record lectures via Zoom at
\url{https://mcgill.zoom.us/j/89579916886?pwd=VlEzYXJFdjBnc1dTbW8rTXNIeWxGZz09} on a \textit{best-effort}
basis.  Sometimes glitches happen so I cannot guarantee all lectures
will be recorded.  If you are watching online and notice I have not
hit the record button, please speak up!\\
\textbf{Update:} Tutorials will be in the piano room on Thursdays at 4
PM.  TA office hours will be on Tuesdays from 11-12 AM, either remote
or (for now) in the TA offices.  Lecturer office hours will be
Tuesdays at 1 PM, mostly remote at
\url{https://mcgill.zoom.us/j/3531521462}.  



\noindent All course materials (other than zoom recordings) will be posted to github at
\url{https://github.com/sievers/phys512-2022/}.  Problem sets will be
submitted via github as well.  Each student should get a github account, and
email their username to the TAs.  Problem sets are due at 11:59 PM on Fridays.

\vskip 0.1in
\noindent In the event of extraordinary circumstances beyond the University's
control, the content and/or evaluation scheme in this course is
subject to change
\vskip 0.1in
\noindent
Instructor information:  Jonathan Sievers (jonathan dot sievers at
mcgill dot ca).  Office: ERP333.  Phone: x2156.
\vskip 0.1in

\noindent
Teaching assistants:  Marcus Merryfield (marcus dot merryfield at
mail dot mcgill dot ca), Rigel Zifkin (rigel dot zifkin at mail dot
mcgill dot ca), and Daniel Coelho (daniel dot coelho at mail dot
mcgill dot ca).  There may be a fourth TA; if so I will post their
information here as well.

\vskip 0.1in
\noindent Topics covered:
\begin{itemize}
\item Brief introduction/review of python and coding practices.  

\item Floating point math/roundoff error.  How to optimize
  {\textit{e.g.}} numerical derivatives.  Resource: ``What every
  computer scientist should know about floating point arithmetic.'' by
  David Goldberg.  

\item Function interpolation and numerical integration.  Error
  analysis of various techniques. 

\item Integration of ODEs. Runge Kutta methods.  Stiff equations, and
  implicit techniques.

\item Linear algebra and linear least-squares fitting.  Singular value
  decomposition and its application to numerically unstable models.
  Legendre and Chebyshev polynomials.  Iterative solutions to large
  problems using conjugate gradient.

\item Nonlienar least-squares fitting.  Newton's method,
  Levenberg-Marquardt, and Markov Chain Monte Carlo.  

\item Discrete Fourier transforms.  Convolutions and applications to
  image processing.  Aliasing and the Nyquist theorem.  Stationary
  noise and matched filters. 

\item Random variables.  Transformation method, rejection method, and
  ratio-of-uniforms.  

\item Partial differential equations. Eulerian and Lagrangian
  techniques.  Advection equation, stability analysis, and the  CFL
  condition.  Inviscid fluid flow, numerical dissipation, and the Lax
  method.  

\item Brief introduction to machine learning.

\end{itemize}

\vskip 0.1in
\noindent
{\textbf{Language Policy}}:\newline
\noindent
In accord with McGill University’s Charter of Student Rights, students
in this course have the right to submit in English or in French any
written work that is to be graded. This does not apply to courses in
which acquiring proficiency in a language is one of the objectives.
\vskip 0.05in
\noindent
Conformément à la Charte des droits de l’étudiant de l’Université
McGill, chaque étudiant a le droit de soumettre en français ou en
anglais tout travail écrit devant être noté, sauf dans le cas des
cours dont l’un des objets est la maîtrise d’une langue. 
\vskip 0.1in

\noindent{\textbf{Academic Integrity:}}\newline
\noindent
McGill University values academic integrity. Therefore, all students
must understand the meaning and consequences of cheating, plagiarism
and other academic offences under the Code of Student Conduct and
Disciplinary Procedures. (Approved by Senate on 29 January 2003) (See
McGill’s guide to academic honesty for more information.)
\vskip 0.05in
\noindent
L'université McGill attache une haute importance à l’honnêteté
académique. Il incombe par conséquent à tous les étudiants de
comprendre ce que l'on entend par tricherie, plagiat et autres
infractions académiques, ainsi que les conséquences que peuvent
avoir de telles actions, selon le Code de conduite de l'étudiant et
des procédures disciplinaires (pour de plus amples renseignements,
veuillez consulter le guide pour l’honnêteté académique de McGill.

\end{document}

